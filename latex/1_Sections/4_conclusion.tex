% Desribe what we achieved (solver)
% What did we find out (Influence of viscosity and shape)
% Relevance of our findings, other interesting questions
In this report we numerically analysed the von Kármán vortex street. We successfully built a solver which leverages Chorin's method to simplify the integration step and breaks it down in smaller steps, which are easier to solve. The method then simplified to treating the advection with a second order Semi-Lagrangian scheme, the diffusion with a second order central difference scheme and the Poisson equation for the pressure with a 5-point stencil method. To achieve great flexibility for the possible shapes inside the domain, we impose the fluid at rest in the form of the shape. Basically, this corresponds to imposing a no-slip condition at the boundary of the shape.

In the second part of our approach we numerically investigated the emergent flow patterns around different objects. First, we exemplary showed the emergence of the von Kármán vortex street in case of a circle placed on the left side of the domain for fluid with intermediate Reynolds numbers. As expected we observe the alternating vortices behind the body. We identified a critical Reynolds number which indicates the dynamics of the flow. For smaller Reynolds numbers the flow is laminar and no vortices emerge. For higher numbers periodic formations emerge, which finally lead to mixing and vortices shredding. Finally, we demonstrated the impact of the shape on the flow. By looking at an airfoil we showed that with intelligent design of the shape the dynamics of the flow can be greatly influenced. With this new shape only with much higher Reynolds numbers we observed vortex shredding and thereby showcased the importance of fluid- and aero-dynamic favorable design. Especially for planes, ships or turbines the shape has a great impact on performance and efficiency, among other essential performance metrics.